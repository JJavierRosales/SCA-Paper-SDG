\documentclass[3p, twocolumn]{elsarticle}


\usepackage[utf8]{inputenc}
\usepackage{lineno,hyperref}
\modulolinenumbers[5]

\journal{International Astronautical Federation (IAF)}

%%%%%%%%%%%%%%%%%%%%%%%
%% Elsevier bibliography styles
%%%%%%%%%%%%%%%%%%%%%%%
%% To change the style, put a % in front of the second line of the current style and
%% remove the % from the second line of the style you would like to use.
%%%%%%%%%%%%%%%%%%%%%%%

%% Numbered
%\bibliographystyle{model1-num-names}

%% Numbered without titles
%\bibliographystyle{model1a-num-names}

%% Harvard
%\bibliographystyle{model2-names.bst}\biboptions{authoryear}

%% Vancouver numbered
%\usepackage{numcompress}\bibliographystyle{model3-num-names}

%% Vancouver name/year
%\usepackage{numcompress}\bibliographystyle{model4-names}\biboptions{authoryear}

%% APA style
%\bibliographystyle{model5-names}\biboptions{authoryear}

%% AMA style
%\usepackage{numcompress}\bibliographystyle{model6-num-names}

%% `Elsevier LaTeX' style
\bibliographystyle{elsarticle-num}
%%%%%%%%%%%%%%%%%%%%%%%


\begin{document}

\begin{frontmatter}

\title{Spacecraft Conjunction Assessment optimization using Deep Learning algorithms applied to Conjunction Data Messages (CDMs)} 



\author{José Javier Rosales Ruiz \corref{cor1}} %\fnref{fn1}}
\ead{j.rosales-ruiz@cranfield.ac.uk} 
% \affiliation[1]{organization={Elsevier B.V.}, addressline={Radarweg 29}, postcode={1043 NX}, city={Amsterdam}, country={The Netherlands}}


\author{Nicola Garzaniti\corref{cor2}} 
\ead{nicola.garzaniti@cranfield.ac.uk}
% \affiliation[2]{organization={Sayahna Foundation}, addressline={JWRA 34, Jagathy}, city={Trivandrum}, postcode={695014}, country={India}}
            
\cortext[cor1]{Corresponding author}
\cortext[cor2]{Co-author}


\begin{abstract}

% In order for an Abstract to be effective when displayed in IEEE {\bfseries\itshape Xplore} as well as through indexing services such as Compendex, INSPEC, Medline, ProQuest, and Web of Science, it must be an accurate, stand-alone reflection of the contents of the article. They shall not contain displayed mathematical equations, numbered reference citations, nor footnotes. They should include three or four different keywords or phrases, as this will help readers to find it. It is important to avoid over-repetition of such phrases as this can result in a page being rejected by search engines. Ensure that your abstract reads well and is grammatically correct.


Since the beginning of the space era, the humankind has not ceased his activity to exploit the potential of the near-Earth space environment. The lack of strict global regulations has lead to a consistent increase of the space density (aggravated by recent collisions in orbit and new megaconstellations of satellites) with Artificial Space Objects (ASOs) which poses a serious threat not only to the sustainability of the  space activity as we know nowadays but also causes a severe environmental damage. 

This increment of space density increases the risk of collisions in orbit which create additional debris triggering a chain reaction, a phenomena commonly known as the \textit{Kessler syndrome}. In an attempt to mitigate this problem, space organizations worldwide have developed Space Situational Awareness (SSA) programmes with integrated Conjunction Assessment systems that rely on processing iteratively space data coming from international Data Sharing schemes and Space Surveillance Tracking systems to identify conjunctions with high risk of collisions. While this approach is still effective, the potential increase of ASOs in orbit expected in the next decade will put any spacecraft collision avoidance system under severe stress if the current technologies used do not evolve fast enough.

The automated data-processing architecture lying within the existing Conjunction Assessment systems follow an iterative and sequential approach where conjunctions with high collision risk probability are identified and continuously re-assessed until Time of Closest Approach (TCA). This re-assessment is performed every time there is new data available coming from either the SST network, SSA data sharing agreements or new ephemeris data from the spacecraft Owners/Operators Flight Dynamics team \cite{Merz-2021-CAServices}\cite{USSPACECOM-2009-SpaceTrack}.

This research project focuses on the data-processing side of the Conjunction Assessment systems' workflow by optimizing the identification and prediction of high collision risk conjunctions by means of Time Series Forecasting (TSF) Deep Learning models. In particular, \textit{Transformer} networks constitute the core of the research for their capability to selectively weight important information from the past\cite{Shi-2022-TSF}.
\end{abstract}

\begin{keyword}
Spacecraft Conjunction Assessment \sep
Space Debris \sep 
Conjunction Data Messages \sep 
Deep Learning \sep 
Recurrent Neural Networks
\end{keyword}

\end{frontmatter}

\linenumbers

%\section{The Elsevier article class}
%
%\paragraph{Installation} If the document class \emph{elsarticle} is not available on your computer, you can download and install the system package \emph{texlive-publishers} (Linux) or install the \LaTeX\ package \emph{elsarticle} using the package manager of your \TeX\ installation, which is typically \TeX\ Live or Mik\TeX.
%
%\paragraph{Usage} Once the package is properly installed, you can use the document class \emph{elsarticle} to create a manuscript. Please make sure that your manuscript follows the guidelines in the Guide for Authors of the relevant journal. It is not necessary to typeset your manuscript in exactly the same way as an article, unless you are submitting to a camera-ready copy (CRC) journal.
%
%\paragraph{Functionality} The Elsevier article class is based on the standard article class and supports almost all of the functionality of that class. In addition, it features commands and options to format the
%\begin{itemize}
%\item document style
%\item baselineskip
%\item front matter
%\item keywords and MSC codes
%\item theorems, definitions and proofs
%\item lables of enumerations
%\item citation style and labeling.
%\end{itemize}
%
%\section{Front matter}
%
%The author names and affiliations could be formatted in two ways:
%\begin{enumerate}[(1)]
%\item Group the authors per affiliation.
%\item Use footnotes to indicate the affiliations.
%\end{enumerate}
%See the front matter of this document for examples. You are recommended to conform your choice to the journal you are submitting to.
%
%\section{Bibliography styles}
%
%There are various bibliography styles available. You can select the style of your choice in the preamble of this document. These styles are Elsevier styles based on standard styles like Harvard and Vancouver. Please use Bib\TeX\ to generate your bibliography and include DOIs whenever available.
%
%Here are two sample references: \cite{Feynman1963118,Dirac1953888}.

%\section*{References}

%\bibliography{refs}

\end{document}