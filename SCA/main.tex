\documentclass[3p, twocolumn]{elsarticle}


\usepackage[utf8]{inputenc}
\usepackage{lineno,hyperref}
\modulolinenumbers[5]

\journal{International Astronautical Federation (IAF)}

%%%%%%%%%%%%%%%%%%%%%%%
%% Elsevier bibliography styles
%%%%%%%%%%%%%%%%%%%%%%%
%% To change the style, put a % in front of the second line of the current style and
%% remove the % from the second line of the style you would like to use.
%%%%%%%%%%%%%%%%%%%%%%%

%% Numbered
%\bibliographystyle{model1-num-names}

%% Numbered without titles
%\bibliographystyle{model1a-num-names}

%% Harvard
%\bibliographystyle{model2-names.bst}\biboptions{authoryear}

%% Vancouver numbered
%\usepackage{numcompress}\bibliographystyle{model3-num-names}

%% Vancouver name/year
%\usepackage{numcompress}\bibliographystyle{model4-names}\biboptions{authoryear}

%% APA style
%\bibliographystyle{model5-names}\biboptions{authoryear}

%% AMA style
%\usepackage{numcompress}\bibliographystyle{model6-num-names}

%% `Elsevier LaTeX' style
\bibliographystyle{elsarticle-num}
%%%%%%%%%%%%%%%%%%%%%%%


\begin{document}

\begin{frontmatter}

\title{Spacecraft Conjunction Assessment optimization using Deep Learning algorithms applied to Conjunction Data Messages (CDMs)} 



\author{José Javier Rosales Ruiz \corref{cor1}} %\fnref{fn1}}
\ead{j.rosales-ruiz@cranfield.ac.uk} 
% \affiliation[1]{organization={Elsevier B.V.}, addressline={Radarweg 29}, postcode={1043 NX}, city={Amsterdam}, country={The Netherlands}}


\author{Nicola Garzaniti\corref{cor2}} 
\ead{nicola.garzaniti@cranfield.ac.uk}
% \affiliation[2]{organization={Sayahna Foundation}, addressline={JWRA 34, Jagathy}, city={Trivandrum}, postcode={695014}, country={India}}
            
\cortext[cor1]{Corresponding author}
\cortext[cor2]{Co-author}


\begin{abstract}

% In order for an Abstract to be effective when displayed in IEEE {\bfseries\itshape Xplore} as well as through indexing services such as Compendex, INSPEC, Medline, ProQuest, and Web of Science, it must be an accurate, stand-alone reflection of the contents of the article. They shall not contain displayed mathematical equations, numbered reference citations, nor footnotes. They should include three or four different keywords or phrases, as this will help readers to find it. It is important to avoid over-repetition of such phrases as this can result in a page being rejected by search engines. Ensure that your abstract reads well and is grammatically correct.

The lack of global regulations on space debris management during the early days of the space era until the last few decades of the $20^{th}$ century resulted in a consistent increase of space debris. Spacecraft collisions in orbit and the industry's growing interest in launching \textit{megaconstellations} of satellites are now exacerbating the problem.

To address those concerns, multiple space organizations worldwide have implemented Space Situational Awareness (SSA) programmes with integrated Conjunction Assessment systems that allows the detection of conjunctions with an \textit{estimated} risk of collision. While this approach has proved to be effective in the last two decades, the foreseen increment of Artificial Space Objects (ASOs) in orbit in the next decade will put any existing Spacecraft Collision Avoidance system under severe stress if the technology does not evolve accordingly to cover the new demands.

This research project focuses on the optimization of the Conjunction Assessment systems with the achievement of two primary objectives; the development of a highly scalable Deep Learning model for Time Series Forecasting (TSF) which is efficient enough to absorb the new data processing workload in a timely manner while improving the accuracy of the existing solutions, and the definition of future design requirements for Space Surveillance Tracking systems to improve the accuracy of observations.

To achieve both milestones, two state-of-the-art \textit{Recurrent Neural Networks} models are used and evaluated in the research project in search of the most optimal solution for their capabilities to selectively weight important information from the past: \textit{Long Short-Term Memory Networks} (LSTMs) and \textit{Transformers}. 

This solution is complemented by the development of a Synthetic Conjunction Data Message Generation (SCDMG) tool with the aim to feed the TSF model with additional virtual data (look-back windows) to address the scarcity of real data open to the public which leads to a poor forecasting performance from the model. A general overview on its limitations and use cases are also part of the scope of the research.

\end{abstract}

\begin{keyword}
Spacecraft Conjunction Assessment \sep
Space Debris \sep 
Conjunction Data Messages \sep 
Deep Learning \sep 
Recurrent Neural Networks
\end{keyword}

\end{frontmatter}

\linenumbers

%\section{The Elsevier article class}
%
%\paragraph{Installation} If the document class \emph{elsarticle} is not available on your computer, you can download and install the system package \emph{texlive-publishers} (Linux) or install the \LaTeX\ package \emph{elsarticle} using the package manager of your \TeX\ installation, which is typically \TeX\ Live or Mik\TeX.
%
%\paragraph{Usage} Once the package is properly installed, you can use the document class \emph{elsarticle} to create a manuscript. Please make sure that your manuscript follows the guidelines in the Guide for Authors of the relevant journal. It is not necessary to typeset your manuscript in exactly the same way as an article, unless you are submitting to a camera-ready copy (CRC) journal.
%
%\paragraph{Functionality} The Elsevier article class is based on the standard article class and supports almost all of the functionality of that class. In addition, it features commands and options to format the
%\begin{itemize}
%\item document style
%\item baselineskip
%\item front matter
%\item keywords and MSC codes
%\item theorems, definitions and proofs
%\item lables of enumerations
%\item citation style and labeling.
%\end{itemize}
%
%\section{Front matter}
%
%The author names and affiliations could be formatted in two ways:
%\begin{enumerate}[(1)]
%\item Group the authors per affiliation.
%\item Use footnotes to indicate the affiliations.
%\end{enumerate}
%See the front matter of this document for examples. You are recommended to conform your choice to the journal you are submitting to.
%
%\section{Bibliography styles}
%
%There are various bibliography styles available. You can select the style of your choice in the preamble of this document. These styles are Elsevier styles based on standard styles like Harvard and Vancouver. Please use Bib\TeX\ to generate your bibliography and include DOIs whenever available.
%
%Here are two sample references: \cite{Feynman1963118,Dirac1953888}.

%\section*{References}

%\bibliography{refs}

\end{document}