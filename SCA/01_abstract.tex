
% In order for an Abstract to be effective when displayed in IEEE {\bfseries\itshape Xplore} as well as through indexing services such as Compendex, INSPEC, Medline, ProQuest, and Web of Science, it must be an accurate, stand-alone reflection of the contents of the article. They shall not contain displayed mathematical equations, numbered reference citations, nor footnotes. They should include three or four different keywords or phrases, as this will help readers to find it. It is important to avoid over-repetition of such phrases as this can result in a page being rejected by search engines. Ensure that your abstract reads well and is grammatically correct.

The lack of global regulations on space debris management during the early days of the space era until the last few decades of the $20^{th}$ century resulted in a consistent increase of space debris. Spacecraft collisions in orbit and the industry's growing interest in launching \textit{megaconstellations} of satellites are now exacerbating the problem.

To address those concerns, multiple space organizations worldwide have implemented Space Situational Awareness (SSA) programmes with integrated Conjunction Assessment systems that allows the detection of conjunctions with an \textit{estimated} risk of collision. While this approach has proved to be effective in the last two decades, the foreseen increment of Artificial Space Objects (ASOs) in orbit in the next decade will put any existing Spacecraft Collision Avoidance (SCA) system under severe stress if the technology does not evolve accordingly to cover the new demands.

This research project focuses on the optimization of the Conjunction Assessment systems with the achievement of two primary objectives; firstly the development of a highly scalable Deep Learning model for Time Series Forecasting (TSF) efficient enough to absorb the new data processing workload in a timely manner while improving the accuracy of the existing solutions. Secondly, the definition of future design requirements for Space Surveillance Tracking systems to improve the accuracy of observations.

To achieve both objectives, two state-of-the-art \textit{Recurrent Neural Networks} models are used and evaluated in this work for their capabilities to selectively weight important information from the past in search of the most optimal solution: \textit{Long Short-Term Memory Networks} (LSTMs) and \textit{Transformers}. 

Additionally, to prevent a poor RNNs forecasting performance driven by the scarcity of real data open to the public, a Synthetic CDM Generation (SCDMG) method is developed with the aim to create additional training virtual CDM data (look-back windows) with which to feed the TSF model. Apart from this benefit, the method also provides the capability to virtually weight uncertainties inherent in the data gathered to evaluate what are the most relevant features upon which most of the effort must be placed in the design of future SST systems to improve the SCA services.
