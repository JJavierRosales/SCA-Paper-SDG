
% In order for an Abstract to be effective when displayed in IEEE {\bfseries\itshape Xplore} as well as through indexing services such as Compendex, INSPEC, Medline, ProQuest, and Web of Science, it must be an accurate, stand-alone reflection of the contents of the article. They shall not contain displayed mathematical equations, numbered reference citations, nor footnotes. They should include three or four different keywords or phrases, as this will help readers to find it. It is important to avoid over-repetition of such phrases as this can result in a page being rejected by search engines. Ensure that your abstract reads well and is grammatically correct.


Since the beginning of the space era, the humankind has not ceased his activity to exploit the potential of the near-Earth space environment. The lack of strict global regulations has lead to a consistent increase of the space density (aggravated by recent collisions in orbit and new megaconstellations of satellites) with Artificial Space Objects (ASOs) which poses a serious threat not only to the sustainability of the  space activity as we know nowadays but also causes a severe environmental damage. 

This increment of space density increases the risk of collisions in orbit which create additional debris triggering a chain reaction, a phenomena commonly known as the \textit{Kessler syndrome}. In an attempt to mitigate this problem, space organizations worldwide have developed Space Situational Awareness (SSA) programmes with integrated Conjunction Assessment systems that rely on processing iteratively space data coming from international Data Sharing schemes and Space Surveillance Tracking systems to identify conjunctions with high risk of collisions. While this approach is still effective, the potential increase of ASOs in orbit expected in the next decade will put any spacecraft collision avoidance system under severe stress if the current technologies used do not evolve fast enough.

The automated data-processing architecture lying within the existing Conjunction Assessment systems follow an iterative and sequential approach where conjunctions with high collision risk probability are identified and continuously re-assessed until Time of Closest Approach (TCA). This re-assessment is performed every time there is new data available coming from either the SST network, SSA data sharing agreements or new ephemeris data from the spacecraft Owners/Operators Flight Dynamics team \cite{Merz-2021-CAServices}\cite{USSPACECOM-2009-SpaceTrack}.

This research project focuses on the data-processing side of the Conjunction Assessment systems' workflow by optimizing the identification and prediction of high collision risk conjunctions by means of Time Series Forecasting (TSF) Deep Learning models. In particular, \textit{Transformer} networks constitute the core of the research for their capability to selectively weight important information from the past\cite{Shi-2022-TSF}.