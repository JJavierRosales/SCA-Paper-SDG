
% In order for an Abstract to be effective when displayed in IEEE {\bfseries\itshape Xplore} as well as through indexing services such as Compendex, INSPEC, Medline, ProQuest, and Web of Science, it must be an accurate, stand-alone reflection of the contents of the article. They shall not contain displayed mathematical equations, numbered reference citations, nor footnotes. They should include three or four different keywords or phrases, as this will help readers to find it. It is important to avoid over-repetition of such phrases as this can result in a page being rejected by search engines. Ensure that your abstract reads well and is grammatically correct.


The scarceness of global regulations on space debris management from the beginning of the space era to the last decades of the $20^{th}$ century has lead to a consistent increase of the space density which is now being aggravated by recent collisions in orbit and the growing interest in the industry to launch megaconstellations of satellites. A higher space density increases the risk of collisions in orbit which in turns can potentially create additional debris in a chain reaction, a phenomena commonly known as the \textit{Kessler syndrome}. As a countermeasure, multiple space organizations worldwide have implemented Space Situational Awareness (SSA) programmes with integrated Conjunction Assessment systems that allows the detection of conjunctions with an \textit{estimated} risk of collision. While this approach has proved to be effective in the last two decades, the foreseen increment of Artificial Space Objects (ASOs) in orbit in the next decade will put any existing Spacecraft Collision Avoidance system under severe stress if the technology does not evolve accordingly to cover the new demands.

This research project focuses on the optimization of the Conjunction Assessment systems through the achievement of two objectives mainly; firstly to develop a Deep Learning model that provides enhanced identification capabilities for high collision risk conjunctions based on past events, and secondly to use this model to identify design requirements for future Space Surveillance Tracking systems to improve the accuracy of collision predictions.  

To achieve both objectives, the use of a Time Series Forecasting (TSF) model available within the Deep Learning field is considered in the present research. In particular, state-of-the-art \textit{Recurrent Neural Networks} such as \textit{Transformers} are placed at the core of the solution proposed for their capabilities to selectively weight important information from the past. 

Complementary to the development of the RNN model, a Synthetic Conjunction Data Message Generation (SCDMG) method is proposed in order to feed the  with additional CDM virtual data to solve the problem of scarcity of real CDMs open to the public. A general overview on its limitations and use cases are also part of the scope of the research.


% with Artificial Space Objects (ASOs) which poses a serious threat not only to the sustainability of the  space activity as we know nowadays but also causes a severe environmental damage. 



% The automated data-processing architecture lying within the existing Conjunction Assessment systems follow an iterative and sequential approach where conjunctions with high collision risk probability are identified and continuously re-assessed until Time of Closest Approach (TCA). This re-assessment is performed every time there is new data available coming from either the SST network, SSA data sharing agreements or new ephemeris data from the spacecraft Owners/Operators Flight Dynamics team \cite{Merz-2021-CAServices}\cite{USSPACECOM-2009-SpaceTrack}.

